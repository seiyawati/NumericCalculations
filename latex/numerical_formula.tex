% 球対象な状況下でナビエストークス方程式(粘性項を切り落としたもの)
\frac{\partial v}{\partial t} + v\frac{\partial v}{\partial r} = -\frac{1}{\rho}\frac{\partial P}{\partial r}

% 速度ポテンシャル
v=\frac{\partial \phi}{\partial r}

% ベルヌーイの式
\frac{\partial }{\partial r}\left[\frac{\partial \phi}{\partial t}(t, r)+\frac{1}{2}\left(\frac{\partial \phi}{\partial r}(t,r)\right)^2+\int_{}^{p(t,r)}\frac{\text{d}p}{\rho} \right] = 0

% 波動方程式
\frac{\partial^2 \phi}{\partial t^2}-c^2\triangle\phi=0

% 波動方程式の解の一つ
\phi(t,r)=\frac{f(t-r/c_{L})}{r}

% 運動学的境界条件
\dot{R}=\frac{\partial \phi}{\partial r}
\dot{R}=\frac{\partial \phi}{\partial r}=ー\frac{f(t-R(t)/c_{L})}{R(t)^2}-\frac{1}{c_{L}}\frac{f'(t-R(t)/c_{L})}{R(t)}

% ベルヌーイの式をrについてR(t)から+∞まで積分した結果
p(t,R(t))=p_{∞}-\rho\left[\frac{\partial \phi}{\partial t}(t, r)+\frac{1}{2}\left(\frac{\partial \phi}{\partial r}(t,r)\right)^2\right]=p_{∞}-\rho\left\{\frac{f'(t-R(t)/c_{L})}{R(t)}+\frac{1}{2}\dot{R(t)^2}\right\}

% ケラー方程式
\left(1-\frac{\dot{R}}{c_{L}}\right)R\ddot{R}+\frac{3}{2}\left(1-\frac{\dot{R}}{3c_{L}}\right)\dot{R}^2 = -\left(1+\frac{\dot{R}}{c_{L}}+\frac{R}{c_{L}}\frac{\text{d}}{\text{dt}}\right)\frac{p_{∞}-p(t, R)}{\rho_{L}}

% 力学的境界条件
P(t,R)=p_{v} + \overline{p_{G}}\left(\frac{\overline{R}}{R}\right)^{3k}-\frac{2\sigma}{R}

% レイリープリセット方程式
R\ddot{R}+\frac{3}{2}\dot{R}^2 + \frac{p_{∞}-p_{B}(R(t))+\frac{2\sigma}{R(t)}}{\rho_{L}} = 0

% 圧力変動分
p_{ij}(t)=-\rho_{L}(\frac{\partial \phi_{j}}{\partial t}+\frac{1}{2}\left(\frac{\partial \phi_{j}}{\partial r}\right)^2)(t,|x_{i}-x_{j}|)

% 相互作用する気泡の運動方程式
\left(1-\frac{\dot{R_{i}}}{c_{∞}}\right)R_{i}\ddot{R_{i}}+\frac{3}{2}\left(1-\frac{\dot{\dot{R}_{i}}}{3c_{∞}}\right)\dot{R_{i}}^2 = \frac{1}{\rho}\left(1+\frac{\dot{R_{i}}}{c_{∞}}\right)\left[p_{B,i}(R_{i})-\left(p_{∞}+\sum_{j \neq i}p_{ij}\right)\right]+\frac{R_{i}}{c_{∞}\rho}\frac{\text{d}}{\text{d}t}\left[p_{B,i}(R_{i}(t))-\sum_{j\neq i}p_{ij}(t)\right]

p_{B,i}(R_{i})=p_{V}+\overline{p_{G,i}}\left(\frac{\overline{R}}{R_{i}}\right)^{3k}-\frac{2\sigma}{R_{i}}

% pijについてin, outを考慮した力学的境界条件
p_i^{out}+\sum_jp_{ij}^{in}=p_{V}+\overline{p_{G,i}}\left(\frac{\overline{R}}{R_{i}}\right)^{3k}-\frac{2\sigma}{R_{i}}

\left(1-\frac{\dot{R}}{c_{∞}}\right)R\ddot{R}+\frac{3}{2}\left(1-\frac{\dot{R}}{3c_{∞}}\right)\dot{R}^2 = -\left(1+\frac{\dot{R}}{c_{∞}}+\frac{R}{c_{∞}}\frac{\text{d}}{\text{dt}}\right)\frac{p_{∞}-p_i^{out}}{\rho_{L}}

% 位相進行速度の式
\frac{\text{d}\theta_{i}}{\text{d}t}(t)=\left(\frac{\text{d}R_{i}}{\text{d}t},\frac{\text{d}V_{i}}{\text{d}t}\right)\mid\left(R_i^{RP}, V_i^{RP}\right)・\left(\dot{R}_i^{RP}, \dot{V}_i^{RP}\right)\frac{1}{(\dot{R}_i^{RP})^2+(\dot{V}_i^{RP})^2}\frac{2\pi}{T_{prd}}\approx\frac{2\pi}{T_{prd}}\left[1-\frac{1}{\rho_{L}}\frac{\dot{V}_i^{RP}}{{R}_i^{RP}}\frac{1}{(\dot{R}_i^{RP})^2+(\dot{V}_i^{RP})^2}\sum_{j\neq i}p_{ij}\right]

% 相互作用入りモデル1周期間での2気泡の位相差の等値線
\int_{0}^{T_{prd}}(\frac{\text{d}\theta_{2}}{\text{d}t}-\frac{\text{d}\theta_{1}}{\text{d}t})(t) dt(\delta\theta_{0}, \frac{\mid x_{2}-x_{1}\mid}{\lambda})

% 相互作用によるエネルギー生成のの等値線
-\int_{0}^{T_{prd}}\sum_{i}V_{i}R_{i}^2(\sum_{j\neq i}p_{ij})(t) dt(\delta\theta_{0}, \frac{\mid x_{2}-x_{1}\mid}{\lambda})

% 位相同調のメカニズム
\frac{\text{d}\theta_{i}}{\text{d}t}(t)\approx\frac{2\pi}{T_{prd}}\left[1-\frac{1}{\rho_{L}}\frac{\dot{V}_i^{RP}}{{R}_i^{RP}}\frac{1}{(\dot{R}_i^{RP})^2+(\dot{V}_i^{RP})^2}\sum_{j\neq i}p_{ij}\right]

\int_{0}^{T_{prd}}(\frac{\text{d}\theta_{2}}{\text{d}t}-\frac{\text{d}\theta_{1}}{\text{d}t})(t)dt